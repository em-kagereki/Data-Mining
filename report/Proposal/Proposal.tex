% interactcadsample.tex
% v1.03 - April 2017

\documentclass[]{interact}

\usepackage{epstopdf}% To incorporate .eps illustrations using PDFLaTeX, etc.
\usepackage{subfigure}% Support for small, `sub' figures and tables
%\usepackage[nolists,tablesfirst]{endfloat}% To `separate' figures and tables from text if required

\usepackage{natbib}% Citation support using natbib.sty
\bibpunct[, ]{(}{)}{;}{a}{}{,}% Citation support using natbib.sty
\renewcommand\bibfont{\fontsize{10}{12}\selectfont}% Bibliography support using natbib.sty

\theoremstyle{plain}% Theorem-like structures provided by amsthm.sty
\newtheorem{theorem}{Theorem}[section]
\newtheorem{lemma}[theorem]{Lemma}
\newtheorem{corollary}[theorem]{Corollary}
\newtheorem{proposition}[theorem]{Proposition}

\theoremstyle{definition}
\newtheorem{definition}[theorem]{Definition}
\newtheorem{example}[theorem]{Example}

\theoremstyle{remark}
\newtheorem{remark}{Remark}
\newtheorem{notation}{Notation}

% see https://stackoverflow.com/a/47122900

% Pandoc citation processing

\usepackage{hyperref}
\usepackage[utf8]{inputenc}
\def\tightlist{}
\usepackage{booktabs}
\usepackage{longtable}
\usepackage{array}
\usepackage{multirow}
\usepackage{wrapfig}
\usepackage{float}
\usepackage{colortbl}
\usepackage{pdflscape}
\usepackage{tabu}
\usepackage{threeparttable}
\usepackage{threeparttablex}
\usepackage[normalem]{ulem}
\usepackage{makecell}
\usepackage{xcolor}

\begin{document}

\articletype{Project proposal (HINF6210-DataMining)}

\title{Predicting mortality in patients with suspected acute cardiac
syndrome using patient demographics and the golden hour interventions.}


\author{\name{Edwin Kagereki$^{}$}
\affil{$^{}$}
}

\thanks{CONTACT Edwin
Kagereki. Email: \href{mailto:kagereki@dal.ca}{\nolinkurl{kagereki@dal.ca}}}

\maketitle



\hypertarget{introduction}{%
\section{Introduction}\label{introduction}}

Cardiovascular diseases (CVDs), principally ischemic heart disease (IHD)
and stroke, are the leading cause of global mortality. In addition to
their increasing global prevalence, they are often associated with poor
survival. \citep{roth}

The concept of the golden hour refers to the vital period by which a
patient with a suspected cardiovascular event should be receiving
definitive treatment to prevent death or irreparable damage to the
heart.Although not set in stone, the chances to save a patient are
usually high if substantive medical attention is given within an hour of
the cardiac event\citep{Okada}.

This project aims at predicting the mortality of patients admitted in
ICU with suspected cardiovascular events using data from patients who
stayed in critical care units.

\hypertarget{dataset-description}{%
\section{Dataset description}\label{dataset-description}}

The MIMIC-III dataset (version 1.4)\citep{johnson} is provided by the
MIT Laboratory of Computational Physiology (LCP) and comprises of
health-related data of patients who stayed in critical care units of the
Beth Israel Deaconess Medical Center, Boston, Massachusetts, between
2001 and 2012. Access to the data was accessed after:

\begin{itemize}
\tightlist
\item
  Completion of a course in protecting human research participants,
  including Health Insurance Portability and Accountability Act (HIPAA)
  requirements.
\item
  Signing a data use agreement, which outlines appropriate data usage
  and security standards, and forbidding efforts to identify individual
  patients.
\end{itemize}

The data is provided as a collection of comma separated value (CSV)
files, along with scripts to help with importing the data into a
relational database. This data was de-identified in accordance with
Health Insurance Portability and Accountability Act (HIPAA) standards
using structured data cleansing and date shifting.\citep{JGoldberger}

\hypertarget{summary-of-data}{%
\subsection{summary of data}\label{summary-of-data}}

\begin{verbatim}
## function (x, ...) 
## {
##     UseMethod("table1")
## }
## <bytecode: 0x000000002dd3b3b0>
## <environment: namespace:table1>
\end{verbatim}

\hypertarget{research-question}{%
\section{Research question}\label{research-question}}

\bibliographystyle{tfcad}
\bibliography{interactcadsample.bib}




\end{document}
